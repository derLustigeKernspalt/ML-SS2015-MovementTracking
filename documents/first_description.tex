\documentclass[11pt,a4paper]{paper}

\usepackage[margin=2cm]{geometry}
\usepackage[utf8]{inputenc}

\usepackage[sort]{natbib}

\title{Activity prediction via movement sensor}
\author{Bakhodir Ashirmatov, Benjamin Quack and Qin Zhao}

\begin{document}


\maketitle

\section{Overview}

Our goal is to predict everyday life activities using a wristband activity tracker equipped 
 with a accelerometer.
We are especially interested in predicting and distinguishing between 
 different computer-related activities.
 
\section{Sensors and data}
 
The predictive data is supposed to consist of the movement information (mainly accelerometric data) 
 measured at the wrist.
Computer-related activities like programming, chatting, text-writing,
 browsing or playing games are to be predicted.
Those activities can easily be monitored by installing programs on the 
 test subjects computer that record the usage and activeness of other 
 application, e.g. browser, word-processing program, text editor for programming 
 and so forth.
Since movement sensors are sensitive and different computer-related activities
 seem to exhibit different movement patterns along the hand-arm region, 
 we are confident that different activities can be
 successfully predicted and distinguished.
 
% [Insert Picture of devise here maybe]
 
\section{Novelty}

Some studies have shown that it is possible to distinguish between different 
 activities like sleeping, walking or exercising using movement information 
 \cite{banosetal2014}.
However to our knowledge, computer-related activities have not been studied yet, 
 at least not exclusively, as well as only with a wristband device.
  
  
\section{Benefits}

Distinguishing between different computer-related activities can be
 useful in a working context. 
It provides employers and researcher with an easy method to examine 
 the different contents included within a particular job, e.g.
 the amount of correspondence versus the amount of coding a programmer 
 spends his time on.
The predictions could also be used for measuring personal productivity 
 and preventing procrastination through intervention.

% Problem: It might be argued that tracking activity on PC is sensless 
% since it is already easy to monitor PC activities via software

\bibliographystyle{plain}
\bibliography{bib/references.bib}

\end{document}